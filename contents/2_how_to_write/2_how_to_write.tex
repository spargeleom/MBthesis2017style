\chapter{論文の書き方}\chaplab{論文の書き方}
本章では, 論文の書き方についての一般的なルールや重要な点を幾つか挙げる. 

\section{句読点}
句読点は「、」「。」ではなく, 「, 」「. 」を使うのが機械学会のあたりの分野での習慣. 全角の「,」「.」か半角+半角スペース「, 」「. 」はどちらでも良さそうだが, 自分の中で統一して書く. 


\section{文字式は斜体, 単位は立体}
\begin{itembox}[c]{間違い}
  \begin{align}
    x = rθ sinφ \quad [m]
  \end{align}
ここでx[m]は変位, θ[rad]は角度とする. 
\end{itembox}
\addtocounter{equation}{-1}
\begin{itembox}[c]{正しくは}
  \begin{align}
    x = r\theta \sin\phi \quad [\mathrm{m}]
  \end{align}
ここで$x[\mathrm{m}]$は変位, $\theta[\mathrm{rad}]$は角度とする. 

\end{itembox}
数式環境中でメートルの単位が斜体に ($m$),  文章中で文字式が立体 (x) になっている. ついでに言うと, $\theta$, $\phi$が全角文字で変換したもの (θ, φ) になっており, \LaTeX では扱うべきではない. また, 三角関数$\sin$や行列式$\det$など汎用的な数学記号についても立体にする. 
ほげほげ
ほげほげ
ほげほげ
ほげほげ
ほげほげ
ほげほげ
ほげほげ
ほげほげ
ほげほげ
ほげほげ
ほげほげ
ほげほげ
ほげほげ
ほげほげ
ほげほげ
ほげほげ
ほげほげ

\section{引用について}
引用箇所には必ず参考文献の参照ラベルを振る\cite{knuth1986texbook}. こんな感じ. 文章や図のコピペは厳禁. 図については, 基本的に自分で描くようにするべき. \LaTeX では\BibTeX を用いるのが良い. 


\section{グラフについて}
